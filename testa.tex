\documentclass[a4paper]{report}
\usepackage[UTF8]{ctex}
\begin{document}
\leftline{\Large\textbf{\kaishu 第一章 \space 场论}}%If you want to reopen dictionary, configuration should be checked
\leftline{\large\textbf{一、标量场 \space 矢量场}}
 对定常的标量场$\varphi(\textbf{r},t)=\varphi_0=constant$为等势线,与之相对应的面就是等势面,等势面上的$\varphi$都相等。函数值改变的主要在等位面的法线方向发生。对于矢量场,主要依靠矢量线来辅助,每一点的切线方向都与该点的矢量方向重合。设$d\textbf{r}$是矢量线的切向元素,定义为
\begin{equation}
     a\times d\textbf{r} =0
\end{equation}
该定义其实有点类似流线的定义。我们可以通过矢量线的疏密程度,来判断矢量的大小。\textit{矢量管定义}\\
\leftline{\large\textbf{1.1、梯度 \space gradient}}
梯度是描写标量场不均匀的性的量度,描写标量场中每点邻域内的函数变化
\begin{equation}
	\frac{\partial\varphi}{\partial s}=\lim_{MM' \to 0}\frac{\varphi(M')-\varphi(M)}{MM'}
\end{equation}而每个方向都有对应的方向导数,但每个方向导数都不是相互独立的,我们可以通过经M点的等位面法线方向n上的方向导数$\frac{\partial \varphi}{\partial n}$通过以下公式
\begin{equation}
	\frac{\partial \varphi}{\partial s}=\frac{\partial \varphi}{\partial n}cos(n,s)
\end{equation}
那么大小为$\frac{\partial \varphi}{\partial n}$,方向为n的失恋称为标量函数$\varphi$的梯度,公式为
\begin{equation}
      grad \varphi=\frac{\partial \varphi}{\partial n} {n} 
\end{equation}
在n方向上导数值最大,即在该方向变化最快,而在等位面的切线方向导数等于0,即不改变。
  定理 若$a=grad\varphi$,且$\varphi$是矢径r的单值函数,则沿任一封闭曲线L的线积分
 \begin{equation*}
 \int \textbf{a}\cdot dr=0
 \end{equation*}
\leftline{\Large\textbf{\kaishu 第二章 \space 流体力学}}
\end{document}
